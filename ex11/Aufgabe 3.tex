\documentclass[a4paper]{article}

\usepackage[ngerman]{babel}
\usepackage[utf8x]{inputenc}
\usepackage{amsmath}
\begin{document}
\subsection*{Aufgabe 2}
Dokumentation des Strategy Patterns in unserer Implementation:\\[0.5cm]
Das handle und die strategies befinden sich im package \textit{...entity.fancy}
 Das handle wird durch das interface \textit{FancyHandle} modelliert. Die ConcreteStrategies, welche dieses implementieren, heißen \textit{ChineseWallStrategy, EarthQuakeStrategy, FloodingStrategy} und \textit{EmptyStrategy}. \textit{FancyHandle} enthält lediglich eine Methode, nämlich \textit{action(int,Player)}. Diese wird von der \textit{Map} aufgerufen und führt das entsprechende event aus. Die \textit{EmptyStrategy} war zwar nicht gefordert, aber ermöglicht es in unserer implementation, auch ohne fancy events zu spielen. Der Client/Context unseres Strategy Patterns ist lediglich die Klasse \textit{Map}. Sie übergibt sich selber, wenn sie die action auslöst.
\subsection*{Aufgabe 3}
	Damit die KI halbwegs fair arbeitet, sollte sie so arbeiten, als gäbe es keine besonderen Kanten. Hierfür würde es in unserer Implementation genügen, die Map-Kopien die beim "Vorausschauen" angelegt werden, ohne Edge mit "Fancy"-Eigenschaften zu erstellen, z.B. mit Map.setFancyID(-1). Weiterhin müsste man für ein besseres Spielerlebnis vermutlich dafür sorgen, dass die AI zu einem bestimmten Anteil Fehlentscheidungen trifft, was man mit Pseudozufall und Normalverteilung implementieren könnte. 
\end{document}