\documentclass[a4paper]{article}

\usepackage[german]{babel}
\usepackage[utf8x]{inputenc}
\usepackage{amsmath}
\begin{document}
\subsection*{Aufgabe 2}
Dokumentation des Strategy Patterns in unserer Implementation:\\[1.0cm]
Das handle und die strategys befinden sich im package  \textit{de.tud.cs.se.ws15.kaesekaestchen_fancy_100_ex11.entity.fancy}. Das handle wird durch das interface \textit{FancyHandle} modelliert. 
\subsection*{Aufgabe 3}
	Damit die KI halbwegs fair arbeitet, sollte sie so arbeiten, als gäbe es keine besonderen Kanten. Hierfür würde es in unserer Implementation genügen, die Map-Kopien die beim "Vorausschauen" angelegt werden, ohne Edge mit "Fancy"-Eigenschaften zu erstellen, z.B. mit Map.setFancyID(-1). Weiterhin müsste man für ein besseres Spielerlebnis vermutlich dafür sorgen, dass die AI zu einem bestimmten Anteil Fehlentscheidungen trifft, was man mit Pseudozufall und Normalverteilung implementieren könnte. 
\end{document}